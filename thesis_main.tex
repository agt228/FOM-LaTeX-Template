%-----------------------------------
% Define document and include general packages
%-----------------------------------
% Tabellen- und Abbildungsverzeichnis stehen normalerweise nicht im
% Inhaltsverzeichnis. Gleiches gilt für das Abkürzungsverzeichnis (siehe unten).
% Manche Dozenten bemängeln das. Die Optionen 'listof=totoc,bibliography=totoc'
% geben das Tabellen- und Abbildungsverzeichnis im Inhaltsverzeichnis (toc=Table
% of Content) aus.
% Da es aber verschiedene Regelungen je nach Dozent geben kann, werden hier
% beide Varianten dargestellt.
\documentclass[12pt,oneside,titlepage,listof=totoc,bibliography=totoc]{scrartcl}
%\documentclass[12pt,oneside,titlepage]{scrartcl}

%-----------------------------------
% Dokumentensprache
%-----------------------------------
%\def\FOMEN{}% Auskommentieren um die Dokumentensprache auf englisch zu ändern
\newif\ifde
\newif\ifen

%-----------------------------------
% Meta informationen
%-----------------------------------
%-----------------------------------
% Meta Informationen zur Arbeit
%-----------------------------------

% Autor
\newcommand{\myAutor}{Fiete Ostkamp, Verena Rakers und Artur Gergert}

% Adresse
\newcommand{\myAdresse}{Breul 23 \\ \> \> \> 48147 Münster}

% Titel der Arbeit
% \newcommand{\myTitel}{Object Detection auf Basis von Satellitenbildern}
\newcommand{\myTitel}{Ansätze zu Optimierung und Automatisierung des OSMI-Index}
\newcommand{\mySubTitel}{}
% Betreuer
\newcommand{\myBetreuer}{Prof. Dr. Rüdiger Buchkremer}

% Lehrveranstaltung
\newcommand{\myLehrveranstaltung}{Big Data Consulting Projekt}

% Matrikelnummer
\newcommand{\myMatrikelNr}{557851, 536491, 562394}

% Ort
\newcommand{\myOrt}{Münster}

% Datum der Abgabe
\newcommand{\myAbgabeDatum}{\today}

% Semesterzahl
\newcommand{\mySemesterZahl}{4}

% Name der Hochschule
\newcommand{\myHochschulName}{FOM Hochschule für Oekonomie \& Management}

% Standort der Hochschule
\newcommand{\myHochschulStandort}{Münster}

% Studiengang
\newcommand{\myStudiengang}{Big Data & Business Analytics}

% Art der Arbeit
\newcommand{\myThesisArt}{Hausarbeit}

% Zu erlangender akademische Grad
\newcommand{\myAkademischerGrad}{Master of Science (M. Sc.)}

% Firma
\newcommand{\myFirma}{}


\ifdefined\FOMEN
%Englisch
\entrue
\usepackage[english]{babel}
\else
%Deutsch
\detrue
\usepackage[ngerman]{babel}
\fi

\newcommand{\langde}[1]{%
   \ifde\selectlanguage{ngerman}#1\fi}
\newcommand{\langen}[1]{%
   \ifen\selectlanguage{english}#1\fi}
\usepackage[utf8]{luainputenc}
\langde{\usepackage[babel,german=quotes]{csquotes}}
\langen{\usepackage[babel,english=british]{csquotes}}
\usepackage[T1]{fontenc}
\usepackage{fancyhdr}
% \usepackage{fancybox}
\usepackage{scrlayer}
\usepackage[a4paper, left=4cm, right=2cm, top=4cm, bottom=2cm]{geometry}
\usepackage{graphicx}
\usepackage{colortbl}
\usepackage[capposition=top]{floatrow}
\usepackage{array}
\usepackage{float}      %Positionierung von Abb. und Tabellen mit [H] erzwingen
\usepackage{footnote}
% Darstellung der Beschriftung von Tabellen und Abbildungen (Leitfaden S. 44)
% singlelinecheck=false: macht die Caption linksbündig (statt zentriert)
% labelfont auf fett: (Tabelle x.y:, Abbildung: x.y)
% font auf fett: eigentliche Bezeichnung der Abbildung oder Tabelle
% Fettschrift laut Leitfaden 2018 S. 45
\usepackage[singlelinecheck=false, labelfont=bf, font=bf]{caption}
\usepackage{caption}
\usepackage{subcaption}
\usepackage{enumitem}
\usepackage{amssymb}
\usepackage{mathptmx}
%\usepackage{minted} %Kann für schöneres Syntax Highlighting genutzt werden. ACHTUNG: Python muss installiert sein.
\usepackage[scaled=0.9]{helvet} % Behebt, zusammen mit Package courier, pixelige Überschriften. Ist, zusammen mit mathptx, dem times-Package vorzuziehen. Details: https://latex-kurs.de/fragen/schriftarten/Times_New_Roman.html
\usepackage{courier}
\usepackage{amsmath}
\usepackage[table]{xcolor}
\usepackage{marvosym}			% Verwendung von Symbolen, z.B. perfektes Eurozeichen

\renewcommand\familydefault{\sfdefault}
\usepackage{ragged2e}

% Mehrere Fussnoten nacheinander mit Komma separiert
\usepackage[hang,multiple]{footmisc}
\setlength{\footnotemargin}{1em}

% todo Aufgaben als Kommentare verfassen für verschiedene Editoren
\usepackage{todonotes}

% Verhindert, dass nur eine Zeile auf der nächsten Seite steht
\setlength{\marginparwidth}{2cm}
\usepackage[all]{nowidow}

%-----------------------------------
% Farbdefinitionen
%-----------------------------------
\definecolor{darkblack}{rgb}{0,0,0}
\definecolor{dunkelgrau}{rgb}{0.8,0.8,0.8}
\definecolor{hellgrau}{rgb}{0.0,0.7,0.99}
\definecolor{mauve}{rgb}{0.58,0,0.82}
\definecolor{dkgreen}{rgb}{0,0.6,0}

%-----------------------------------
% Pakete für Tabellen
%-----------------------------------
\usepackage{epstopdf}
\usepackage{nicefrac} % Brüche
\usepackage{multirow}
\usepackage{rotating} % vertikal schreiben
\usepackage{mdwlist}
\usepackage{tabularx}% für Breitenangabe

%-----------------------------------
% sauber formatierter Quelltext
%-----------------------------------
\usepackage{listings}
% JavaScript als Sprache definieren:
\lstdefinelanguage{JavaScript}{
	keywords={break, super, case, extends, switch, catch, finally, for, const, function, try, continue, if, typeof, debugger, var, default, in, void, delete, instanceof, while, do, new, with, else, return, yield, enum, let, await},
	keywordstyle=\color{blue}\bfseries,
	ndkeywords={class, export, boolean, throw, implements, import, this, interface, package, private, protected, public, static},
	ndkeywordstyle=\color{darkgray}\bfseries,
	identifierstyle=\color{black},
	sensitive=false,
	comment=[l]{//},
	morecomment=[s]{/*}{*/},
	commentstyle=\color{purple}\ttfamily,
	stringstyle=\color{red}\ttfamily,
	morestring=[b]',
	morestring=[b]"
}

\lstset{
	%language=JavaScript,
	numbers=left,
	numberstyle=\tiny,
	numbersep=5pt,
	breaklines=true,
	showstringspaces=false,
	frame=l ,
	xleftmargin=5pt,
	xrightmargin=5pt,
	basicstyle=\ttfamily\scriptsize,
	stepnumber=1,
	keywordstyle=\color{blue},          % keyword style
  	commentstyle=\color{dkgreen},       % comment style
  	stringstyle=\color{mauve}         % string literal style
}

%-----------------------------------
%Literaturverzeichnis Einstellungen
%-----------------------------------

% Biblatex

\usepackage{url}
\urlstyle{same}

%%%% Neuer Leitfaden (2018)
\usepackage[
backend=biber,
style=ext-authoryear-ibid, % Auskommentieren und nächste Zeile einkommentieren, falls "Ebd." (ebenda) nicht für sich-wiederholende Fussnoten genutzt werden soll.
%style=ext-authoryear,
maxcitenames=3,	% mindestens 3 Namen ausgeben bevor et. al. kommt
maxbibnames=999,
mergedate=false,
date=iso,
seconds=true, %werden nicht verwendet, so werden aber Warnungen unterdrückt.
urldate=iso,
innamebeforetitle,
dashed=false,
autocite=footnote,
doi=false,
useprefix=true, % 'von' im Namen beachten (beim Anzeigen)
mincrossrefs = 1
]{biblatex}%iso dateformat für YYYY-MM-DD

%weitere Anpassungen für BibLaTex
\input{skripte/modsBiblatex2018}

%%%%% Alter Leitfaden. Ggf. Einkommentieren und Bereich hierüber auskommentieren
%\usepackage[
%backend=biber,
%style=numeric,
%citestyle=authoryear,
%url=false,
%isbn=false,
%notetype=footonly,
%hyperref=false,
%sortlocale=de]{biblatex}

%weitere Anpassungen für BibLaTex
%\input{skripte/modsBiblatex}

%%%% Ende Alter Leitfaden

%Bib-Datei einbinden
\addbibresource{literatur/literatur.bib}

% Zeilenabstand im Literaturverzeichnis ist Einzeilig
% siehe Leitfaden S. 14
\AtBeginBibliography{\singlespacing}

%-----------------------------------
% Silbentrennung
%-----------------------------------
\usepackage{hyphsubst}
\HyphSubstIfExists{ngerman-x-latest}{%
\HyphSubstLet{ngerman}{ngerman-x-latest}}{}

%-----------------------------------
% Pfad fuer Abbildungen
%-----------------------------------
\graphicspath{{./}{./abbildungen/}}

%-----------------------------------
% Weitere Ebene einfügen
%-----------------------------------
\input{skripte/weitereEbene}

%-----------------------------------
% Paket für die Nutzung von Anhängen
%-----------------------------------
\usepackage{appendix}

%-----------------------------------
% Zeilenabstand 1,5-zeilig
%-----------------------------------
\usepackage{setspace}
\onehalfspacing

%-----------------------------------
% Absätze durch eine neue Zeile
%-----------------------------------
\setlength{\parindent}{0mm}
\setlength{\parskip}{0.8em plus 0.5em minus 0.3em}

\sloppy					%Abstände variieren
\pagestyle{headings}

%----------------------------------
% Präfix in das Abbildungs- und Tabellenverzeichnis aufnehmen, statt nur der Nummerierung (siehe Issue #206).
%----------------------------------
\KOMAoption{listof}{entryprefix} % Siehe KOMA-Script Doku v3.28 S.153
\BeforeStartingTOC[lof]{\renewcommand*\autodot{:}} % Für den Doppelpunkt hinter Präfix im Abbildungsverzeichnis
\BeforeStartingTOC[lot]{\renewcommand*\autodot{:}} % Für den Doppelpunkt hinter Präfix im Tabellenverzeichnis

%-----------------------------------
% Abkürzungsverzeichnis
%-----------------------------------
\usepackage[printonlyused]{acronym}

%-----------------------------------
% Symbolverzeichnis
%-----------------------------------
% Quelle: https://www.namsu.de/Extra/pakete/Listofsymbols.pdf
\usepackage[final]{listofsymbols}

%-----------------------------------
% Glossar
%-----------------------------------
\usepackage{glossaries}
\glstoctrue %Auskommentieren, damit das Glossar nicht im Inhaltsverzeichnis angezeigt wird.
\makenoidxglossaries
\input{abkuerzungen/glossar}

%-----------------------------------
% PDF Meta Daten setzen
%-----------------------------------
\usepackage[hyperfootnotes=false]{hyperref} %hyperfootnotes=false deaktiviert die Verlinkung der Fußnote. Ansonsten inkompaibel zum Paket "footmisc"
% Behebt die falsche Darstellung der Lesezeichen in PDF-Dateien, welche eine Übersetzung besitzen
% siehe Issue 149
\makeatletter
\pdfstringdefDisableCommands{\let\selectlanguage\@gobble}
\makeatother

\hypersetup{
    pdfinfo={
        Title={\myTitel},
        Subject={\myStudiengang},
        Author={\myAutor},
        Build=1.1
    }
}

%-----------------------------------
% PlantUML
%-----------------------------------
%\usepackage{plantuml}

%-----------------------------------
% Umlaute in Code korrekt darstellen
% siehe auch: https://en.wikibooks.org/wiki/LaTeX/Source_Code_Listings
%-----------------------------------
\lstset{literate=
	{á}{{\'a}}1 {é}{{\'e}}1 {í}{{\'i}}1 {ó}{{\'o}}1 {ú}{{\'u}}1
	{Á}{{\'A}}1 {É}{{\'E}}1 {Í}{{\'I}}1 {Ó}{{\'O}}1 {Ú}{{\'U}}1
	{à}{{\`a}}1 {è}{{\`e}}1 {ì}{{\`i}}1 {ò}{{\`o}}1 {ù}{{\`u}}1
	{À}{{\`A}}1 {È}{{\'E}}1 {Ì}{{\`I}}1 {Ò}{{\`O}}1 {Ù}{{\`U}}1
	{ä}{{\"a}}1 {ë}{{\"e}}1 {ï}{{\"i}}1 {ö}{{\"o}}1 {ü}{{\"u}}1
	{Ä}{{\"A}}1 {Ë}{{\"E}}1 {Ï}{{\"I}}1 {Ö}{{\"O}}1 {Ü}{{\"U}}1
	{â}{{\^a}}1 {ê}{{\^e}}1 {î}{{\^i}}1 {ô}{{\^o}}1 {û}{{\^u}}1
	{Â}{{\^A}}1 {Ê}{{\^E}}1 {Î}{{\^I}}1 {Ô}{{\^O}}1 {Û}{{\^U}}1
	{œ}{{\oe}}1 {Œ}{{\OE}}1 {æ}{{\ae}}1 {Æ}{{\AE}}1 {ß}{{\ss}}1
	{ű}{{\H{u}}}1 {Ű}{{\H{U}}}1 {ő}{{\H{o}}}1 {Ő}{{\H{O}}}1
	{ç}{{\c c}}1 {Ç}{{\c C}}1 {ø}{{\o}}1 {å}{{\r a}}1 {Å}{{\r A}}1
	{€}{{\EUR}}1 {£}{{\pounds}}1 {„}{{\glqq{}}}1
}

%-----------------------------------
% Kopfbereich / Header definieren
%-----------------------------------
\pagestyle{fancy}
\fancyhf{}
% Seitenzahl oben, mittig, mit Strichen beidseits
% \fancyhead[C]{-\ \thepage\ -}

% Seitenzahl oben, mittig, entsprechend Leitfaden ohne Striche beidseits
\fancyhead[C]{\thepage}
%\fancyhead[L]{\leftmark}							% kein Footer vorhanden
% Waagerechte Linie unterhalb des Kopfbereiches anzeigen. Laut Leitfaden ist
% diese Linie nicht erforderlich. Ihre Breite kann daher auf 0pt gesetzt werden.
\renewcommand{\headrulewidth}{0.4pt}
%\renewcommand{\headrulewidth}{0pt}

%-----------------------------------
% Damit die hochgestellten Zahlen auch auf die Fußnote verlinkt sind (siehe Issue 169)
%-----------------------------------
\hypersetup{colorlinks=true, breaklinks=true, linkcolor=darkblack, citecolor=darkblack, menucolor=darkblack, urlcolor=darkblack, linktoc=all, bookmarksnumbered=false, pdfpagemode=UseOutlines, pdftoolbar=true}
\urlstyle{same}%gleiche Schriftart für den Link wie für den Text

%-----------------------------------
% Start the document here:
%-----------------------------------
\begin{document}

\pagenumbering{Roman}								% Seitennumerierung auf römisch umstellen
\newcolumntype{C}{>{\centering\arraybackslash}X}	% Neuer Tabellen-Spalten-Typ:
%Zentriert und umbrechbar

%-----------------------------------
% Textcommands
%-----------------------------------
\input{skripte/textcommands}

%-----------------------------------
% Titlepage
%-----------------------------------
\begin{titlepage}
    \newgeometry{left=2cm, right=2cm, top=2cm, bottom=2cm}
    \begin{center}
        \includegraphics[width=2.3cm]{abbildungen/fomLogo} \\
        \vspace{.5cm}
        \begin{Large}\textbf{\myHochschulName}\end{Large}\\
        \vspace{.5cm}
        \begin{Large}\langde{Hochschulzentrum}\langen{university location} \myHochschulStandort\end{Large}\\
        \vspace{2cm}
        \begin{Large}\textbf{\myThesisArt}\end{Large}\\
        \vspace{.5cm}
        % \langde{Berufsbegleitender Studiengang}
        % \langen{part-time degree program}\\
        % \mySemesterZahl. Semester\\
        \langde{im Studiengang}\langen{in the study course} \myStudiengang
        \vspace{1.7cm}

        \langde{zur Erlangung des Grades eines}\langen{to obtain the degree of}\\
        \vspace{0.5cm}
        \begin{Large}{\myAkademischerGrad}\end{Large}\\
        % Oder für Hausarbeiten:
        %\textbf{im Rahmen der Lehrveranstaltung}\\
        %\textbf{\myLehrveranstaltung}\\
        \vspace{1.8cm}
        \langde{über das Thema}
        \langen{on the subject}\\
        \vspace{0.5cm}
        \large{\textbf{\myTitel}}\\
        \vspace{2cm}
        \langde{von}\langen{by}\\
        \vspace{0.5cm}
        \begin{Large}{\myAutor}\end{Large}\\
    \end{center}
    \normalsize
    \vfill
    \begin{tabular}{ l l }
        \langde{Betreuer} % für Hausarbeiten
        %\langde{Erstgutachter} % für Bachelor- / Master-Thesis
        \langen{Advisor}: & \myBetreuer\\
        \langde{Matrikelnummer}
        \langen{Matriculation Number}: & \myMatrikelNr\\
        \langde{Abgabedatum}
        \langen{Submission}: & \myAbgabeDatum
        \\
    \end{tabular}
\end{titlepage}


%-----------------------------------
% Vorwort (optional; bei Verwendung beide Zeilen entkommentieren und unter Inhaltsverzeichnis setcounter entsprechend anpassen)
%-----------------------------------
%\section*{Vorwort}
Bei Bedarf erscheint vor dem Inhaltsverzeichnis ein Vorwort. Es erhält keine Kapitelnummer und wird nicht im Inhaltsverzeichnis aufgeführt (Auszug aus dem Leitfaden zur Gestaltung wissenschaftlicher Arbeiten / Dekanat ING \& IT Management, Februar 2022).

Lorem ipsum dolor sit amet, consetetur sadipscing elitr, sed diam nonumy eirmod tempor invidunt ut labore et dolore magna aliquyam erat, sed diam voluptua. At vero eos et accusam et justo duo dolores et ea rebum. Stet clita kasd gubergren, no sea takimata sanctus est Lorem ipsum dolor sit amet. Lorem ipsum dolor sit amet, consetetur sadipscing elitr, sed diam nonumy eirmod tempor invidunt ut labore et dolore magna aliquyam erat, sed diam voluptua. At vero eos et accusam et justo duo dolores et ea rebum. Stet clita kasd gubergren, no sea takimata sanctus est Lorem ipsum dolor sit amet. Lorem ipsum dolor sit amet, consetetur sadipscing elitr, sed diam nonumy eirmod tempor invidunt ut labore et dolore magna aliquyam erat, sed diam voluptua. At vero eos et accusam et justo duo dolores et ea rebum. Stet clita kasd gubergren, no sea takimata sanctus est Lorem ipsum dolor sit amet. 

Duis autem vel eum iriure dolor in hendrerit in vulputate velit esse molestie consequat, vel illum dolore eu feugiat nulla facilisis at vero eros et accumsan et iusto odio dignissim qui blandit praesent luptatum zzril delenit augue duis dolore te feugait nulla facilisi. Lorem ipsum dolor sit amet, consectetuer adipiscing elit, sed diam nonummy nibh euismod tincidunt ut laoreet dolore magna aliquam erat volutpat. 
\\[1cm]
{\myOrt}, März 2022

{\myAutor}
%\newpage

%-----------------------------------
% Inhaltsverzeichnis
%-----------------------------------
% Um das Tabellen- und Abbbildungsverzeichnis zu de/aktivieren ganz oben in Documentclass schauen
\setcounter{page}{2}
\addtocontents{toc}{\protect\enlargethispage{-20mm}}% Die Zeile sorgt dafür, dass das Inhaltsverzeichnisseite auf die zweite Seite gestreckt wird und somit schick aussieht. Das sollte eigentlich automatisch funktionieren. Wer rausfindet wie, kann das gern ändern.
\setcounter{tocdepth}{4}
\tableofcontents
\newpage

%-----------------------------------
% Abbildungsverzeichnis
%-----------------------------------
\listoffigures
\newpage
%-----------------------------------
% Tabellenverzeichnis
%-----------------------------------
\listoftables
\newpage
%-----------------------------------
% Abkürzungsverzeichnis
%-----------------------------------
% Falls das Abkürzungsverzeichnis nicht im Inhaltsverzeichnis angezeigt werden soll
% dann folgende Zeile auskommentieren.
\addcontentsline{toc}{section}{\abbreHeadingName}

\section*{\langde{Abkürzungsverzeichnis}\langen{List of Abbreviations}}

\begin{acronym}[WYSIWYG]\itemsep0pt %der Parameter in Klammern sollte die längste Abkürzung sein. Damit wird der Abstand zwischen Abkürzung und Übersetzung festgelegt
  \acro{CNN}{Convolutional Neural Network}
  \acro{FN}{False Negative}
  \acro{FP}{False Positive}
  \acro{FP}{False Positive}
  \acro{NER}{Named Entity Recognition}
  \acro{NLP}{Natural Language Processing}
  \acro{OSMI}{Online Sensory Marketing Index}
  \acro{TP}{True Positive}
  \acro{TF-IDF}{Term Frequency-Inverse Document Frequency}
\end{acronym}
\newpage

%-----------------------------------
% Symbolverzeichnis
%-----------------------------------
% In Overleaf führt der Einsatz des Symbolverzeichnisses zu einem Fehler, der aber ignoriert werdne kann
% Falls das Symbolverzeichnis nicht im Inhaltsverzeichnis angezeigt werden soll
% dann folgende Zeile auskommentieren.
\addcontentsline{toc}{section}{\symheadingname}
\input{skripte/symbolDef}
\listofsymbols
\newpage

%-----------------------------------
% Glossar
%-----------------------------------
\printnoidxglossaries
\newpage

%-----------------------------------
% Sperrvermerk
%-----------------------------------
%\input{kapitel/anhang/sperrvermerk}

%-----------------------------------
% Seitennummerierung auf arabisch und ab 1 beginnend umstellen
%-----------------------------------
\pagenumbering{arabic}
\setcounter{page}{1}

%-----------------------------------
% Kapitel / Inhalte
%-----------------------------------
% Die Kapitel werden über folgende Datei eingebunden
\input{skripte/kapitelUebersicht.tex}

%-----------------------------------
% Apendix / Anhang
%-----------------------------------
\newpage
\section*{\AppendixName} %Überschrift "Anhang", ohne Nummerierung
\addcontentsline{toc}{section}{\AppendixName} %Den Anhang ohne Nummer zum Inhaltsverzeichnis hinzufügen

\begin{appendices}
% Nachfolgende Änderungen erfolgten aufgrund von Issue 163
\makeatletter
\renewcommand\@seccntformat[1]{\csname the#1\endcsname:\quad}
\makeatother
\addtocontents{toc}{\protect\setcounter{tocdepth}{0}} %
	\renewcommand{\thesection}{\AppendixName\ \arabic{section}}
	\renewcommand\thesubsection{\AppendixName\ \arabic{section}.\arabic{subsection}}
	\section{Beispielanhang}\label{Beispielanhang}
% Dieser Abschnitt dient nur dazu zu demonstrieren, wie ein Anhang aufgebaut seien kann.
% \subsection{Weitere Gliederungsebene}
% Auch eine zweite Gliederungsebene ist möglich.
% \section{Bilder}
% Auch mit Bildern.
% Diese tauchen nicht im Abbildungsverzeichnis auf.
% \begin{figure}[H]
%     \centering
%     \caption[]{Beispielbild}
% 	\label{fig:Beispielbild}
%     \includegraphics[width=1\textwidth]{verzeichnisStruktur}
% \end{figure}
\end{appendices}
\addtocontents{toc}{\protect\setcounter{tocdepth}{2}}

%-----------------------------------
% Literaturverzeichnis
%-----------------------------------
\newpage

% Die folgende Zeile trägt ALLE Werke aus literatur.bib in das
% Literaturverzeichnis ein, egal ob sie zietiert wurden oder nicht.
% Der Befehl ist also nur zum Test der Skripte sinnvoll und muss bei echten
% Arbeiten entfernt werden.
%\nocite{*}

%\addcontentsline{toc}{section}{Literatur}

% Die folgenden beiden Befehle würden ab dem Literaturverzeichnis wieder eine
% römische Seitennummerierung nutzen.
% Das ist nach dem Leitfaden nicht zu tun. Dort steht nur dass 'sämtliche
% Verzeichnisse VOR dem Textteil' römisch zu nummerieren sind. (vgl. S. 3)
%\pagenumbering{Roman} %Zähler wieder römisch ausgeben
%\setcounter{page}{4}  %Zähler manuell hochsetzen

% Ausgabe des Literaturverzeichnisses

% Keine Trennung der Werke im Literaturverzeichnis nach ihrer Art
% (Online/nicht-Online)
%\begin{RaggedRight}
%\printbibliography
%\end{RaggedRight}

% Alternative Darstellung, die laut Leitfaden genutzt werden sollte.
% Dazu die Zeilen auskommentieren und folgenden code verwenden:

% Literaturverzeichnis getrennt nach Nicht-Online-Werken und Online-Werken
% (Internetquellen).
% Die Option nottype=online nimmt alles, was kein Online-Werk ist.
% Die Option heading=bibintoc sorgt dafür, dass das Literaturverzeichnis im
% Inhaltsverzeichnis steht.
% Es ist übrigens auch möglich mehrere type- bzw. nottype-Optionen anzugeben, um
% noch weitere Arten von Zusammenfassungen eines Literaturverzeichnisse zu
% erzeugen.
% Beispiel: [type=book,type=article]
\printbibliography[nottype=online,heading=bibintoc,title={\langde{Literaturverzeichnis}\langen{Bibliography}}]

% neue Seite für Internetquellen-Verzeichnis
\newpage

% Laut Leitfaden 2018, S. 14, Fussnote 44 stehen die Internetquellen NICHT im
% Inhaltsverzeichnis, sondern gehören zum Literaturverzeichnis.
% Die Option heading=bibintoc würde die Internetquelle als eigenen Eintrag im
% Inhaltsverzeicnis anzeigen.
%\printbibliography[type=online,heading=bibintoc,title={\headingNameInternetSources}]
\printbibliography[type=online,heading=subbibliography,title={\headingNameInternetSources}]

\input{kapitel/anhang/erklaerung}
\end{document}
