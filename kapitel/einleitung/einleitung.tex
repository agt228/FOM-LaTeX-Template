\section{Einleitung}
Für Unternehmen wird es immer wichtiger die potenziellen Kunden gezielt anzusprechen und zum Kauf anzuregen. Eine Möglichkeit kann hierbei das multisensorische Marketing bieten, welches zur Beeinflussung des Kaufverhaltens mehrere Sinne anspricht.
ine Umfrage der Mood Media Corporation \footcite[\vglf][\pagef 53]{steiner2022}, die unter 10.000 Menschen weltweit durchgeführt wurde, hat ergeben, dass 75\% der Kunden sich länger in einem Geschäft aufhalten würden, wenn durch Musik oder Düfte eine positive Atmosphäre geschaffen wird.\footcite[\vglf][\pagef 6]{moodmedia2019}
Multisensorisches Marketing anzuwenden kann demnach einen positiven Effekt auf das Kaufverhalten haben.

\subsection{Problemstellung}
Die Anwendung von multisensorischem Marketing findet zurzeit mehrheitlich in stationären Geschäften statt. Da der Online-Handel weiterhin an Wachstum gewinnt sollte auch hier geprüft werden inwiefern multisensorisches Marketing eingesetzt werden kann.
Ein von K. Hamacher entwickelter Index, der Online Sensory Marketing Index \acl{OSMI} setzt an dieser Thematik an. Es ist ein Evaluationsmodell, welches Webseiten anhand ihrer
multisensuelle Kommunikationsqualitäten bewertet. Allerdings bietet der \acl{OSMI} noch Raum für Optimierungen. So wird der \acl{OSMI} bisher manuell und subjektiv ermittelt und die Ergebnisse lediglich in einer Tabelle dargestellt.
Des Weiteren wird bei der Beurteilung einer Webseite nicht nach Produktwebseiten und nicht-Produktseiten unterschieden und die Textanalyse mittels TF-IDF analysiert die sensorischen Wörter nicht im Kontext des gesamten Satzes.

\subsection{Zielsetzung}
Aus der Problemstellung lässt sich daher nachfolgende Zielsetzung für diese Ausarbeitung ableiten. Es soll eine Webseite erstellt werden, in derer unterschiedliche Produktwebseiten automatisiert hinsichtlich ihrer textlichen Ansprache an die Sinne des Konsumenten aufgeführt werden.
Durch eine Evaluierung wird dargelegt wie gut oder wie schlecht die Webseite die Sinne Tasten, Riechen, Hören, Schmecken und Sehen anspricht.
Darauf aufbauend könnten dann die Webseiten-Betreiber die Webseiten optimieren, um die Kunden besser anzusprechen.

\subsection{Vorgehen}
Ein erster Schritt in der Umsetzung der Zielsetzung ist es die relevanten produktspezifischen Webseiten herauszufiltern. Mittels einer Bildklassifizierung soll dies umgesetzt werden, um diese für die weitere Bearbeitung verwenden zu können.
Andere Unterseiten zum Unternehmen oder zu Karrieremöglichkeiten fließen somit nicht mehr in eine Bewertung ein. Im zweiten Schritt werden durch \acf{NLP} die Texte auf den Produktseiten analysiert. Es werden gezielt Wörter - unter Einbezug des Satzkontextes - markiert, die auf eines der fünf Sinne hindeuten.
Eine Beurteilung dieses Ergebnisses wird anhand eines Benchmarkings ermittelt und auf einer Webseite zur Verfügung gestellt.

