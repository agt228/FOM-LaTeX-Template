\section{Einleitung} \label{sec: Einleitung}

\subsection{Motivation} \label{sec: Motivation}

% todo Einige Quellen einfügen

Der Ertragsdruck der Banken in Deutschland ist in den letzten Jahren deutlich gestiegen.
Dies ist unter anderem dem Niedrigzinsphase geschuldet, in der sich Europa in den letzten Jahren befindet und zudem
dem Eintritt neuer Herausforderer im Bankensektor wie beispielsweise diverser sogenannter Neo-Banken.
Des Weiteren sind die regulatorischen Anforderungen für das Eigenkapital der Banken in den letzten Jahren massiv erhöht worden.
Aus den zuvor angeführten Gründen ist es notwendig zum einen die Genauigkeit der Ertragsplanung zu erhöhen, um
gegenüber der Aufsicht nachweisen zu können, dass regulatorische Anforderungen an das Kapital der Bank auch in Zukunft
eingehalten werden können.
Genau so essentiell wie eine realistische Ertragsplanung sind auch die Erträge an sich, damit die Bank auch in Zukunft
rentabel bleibt und am Markt bestehen bleiben kann.\\

Die Zinsstruktur am Geld- und Kapitalmarkt ist mitunter ein zentraler Faktor für die Erträge und Aufwendungen, die eine
Bank erwirtschaftet.
Jene stellt den Preis des Geldes in Abhängigkeit von der Laufzeit dar.
Es gibt unterschiedliche Zinsstrukturkurven am Markt, die sich beispielsweise von Land und/oder der jeweiligen Währung
unterscheiden.\\

In der Vergangenheit wurden bereits einige Versuche unternommen, die Entwicklung der Zinsstrukturkurve vorherzusagen.
Diese Versuche beruhten auf Expertenschätzungen und sind meist jedoch nicht valide gewesen und führten somit unter anderem zu
hohen Soll-Ist-Abweichungen bei den Zinserträgen der Bank.
Da sich mittlerweile eine Vielzahl an Werkzeugen im Bereich der künstlichen Intelligenz am Markt etabliert haben, die
solche Vorhersagen unterstützen, sollten Unternehmen hiervon Gebrauch machen.

\subsection{Zielsetzung} \label{sec: Zielsetzung}

Die Zielsetzung dieser Arbeit liegt darin, den Business Analytics Prozess, wie er von Seiter beschrieben worden ist, an
einer praktischen Problemstellung anzuwenden, um für diese eine mögliche Lösung zu finden.
Im Rahmen der Arbeit soll ein Modell entwickelt werden, welches die Prognose der Zinsstrukturkurve ermöglicht.
Dabei liegt das Ziel nicht das optimalste Modell zu entwickeln, sondern viel mehr darauf den Analytics Prozess sauber
anzuwenden, wenngleich der Autor dieser Hausarbeit die Ambition verfolgt ein aussagekräftiges Ergebnis zu erzielen.

\subsection{Vorgehen} \label{sec: Vorgehen}
Der Aufbau der vorliegen Hausarbeit orientiert sich stringet an dem vom Seiter aufgezeigten Vorgehen eines Businesss
Analytics Prozesses.
In Kapitel~\ref{sec: Methodik} werden die theoretischen Konzepte der Teilprozessschritte
\begin{enumerate}
    \item Framing
    \item Allocation
    \item Analytics
    \item Preparation
\end{enumerate}
kurz zusammenfassend beschrieben.~\footcite[\vglf][\pagef 12]{seiter.2019}
Anschließend werden jene Teilprozessschritte anhand des gewählten Praxisbeispiels im Kapitel~\ref{sec: Umsetzung}
umgesetzt und beschrieben.
Den Schlussteil der Arbeit bildet das~\fullref{sec: Fazit}, in dem eine abschließende Betrachtung und Bewertung des
umgesetzten Business Analytics Prozesses aufgestellt wird, sowie mögliche Verbesserungs- und Weiterentwicklungspotentiale
der Ergebnisse aufgezeigt werden.
