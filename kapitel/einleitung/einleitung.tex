\section{Einleitung}  \textcolor{red}{QUELLEN} \\
Für Unternehmen wird es immer wichtiger die potenziellen Kunden gezielt anzusprechen und zum Kauf anzuregen. Eine Möglichkeit kann hierbei das multisensorische Marketing bieten, welches mehrere Sinne der Kunden anspricht, um das Kaufverhalten zu beeinflussen. Eine Umfrage der Mood Media Corporation , die unter 10k Menschen weltweit durchgeführt wurde, hat ergeben, dass 75\% der Kunden sich länger in einem Geschäft aufhalten würden, wenn durch Musik oder Düfte eine positive Atmosphäre geschaffen wird.
Multisensorisches Marketing anzuwenden kann demnach einen positiven Effekt auf das Unternehmen haben.

\subsection{Problemstellung}
\textcolor{red}{ÜBERARBEITEN \& OSMI irgendwo mit einbauen} \\
Die Anwendung von Multisensorischem Marketing findet zurzeit mehrheitlich in stationären Geschäften statt. Da der Online-Handel weiterhin an Wachstum gewinnt sollte auch hier geprüft werden inwiefern multisens. Marketing eingesetzt werden kann.
Bisher beschränkt sich die Anwendung im E-Commerce auf den Einsatz von Musik und visuelle Elemente.

\subsection{Zielsetzung}
Aus der Problemstellung lässt sich daher nachfolgende Zielsetzung für diese Ausarbeitung ableiten. Es soll eine Webseite erstellt werden, in derer unterschiedliche Produktwebseiten hinsichtlich ihrer textlichen Ansprache an die Sinne des Konsumenten aufgeführt werden. Durch eine Evaluierung wird dargelegt wie gut oder wie schlecht die Webseite die Sinne Tasten, Riechen, Hören, Schmecken und Sehen anspricht.
Darauf aufbauend könnten dann die Webseiten-Betreiber die Webseiten optimieren, um die Kunden besser anzusprechen.

\subsection{Vorgehen}
Ein erster Schritt in der Umsetzung der Zielsetzung ist es die relevanten Produktspezifischen Webseiten herauszufiltern. Mittels einer Bildklassifizierung sollen Produktwebseiten klassifiziert werden, um diese für die weitere Bearbeitung verwenden zu können. Andere Unterseiten zum Unternehmen oder zu Karrieremöglichkeiten fließen somit nicht mehr in eine Bewertung ein. Im zweiten Schritt werden durch Natural Language Processing die Texte auf den Produktseiten analysiert. Es werden gezielt Wörter und deren Umgebung, die auf eines der fünf Sinne deuten markiert.
Eine Beurteilung dieses Ergebnisses wird anhand eins Benchmarkings ermittelt und auf einer Webseite zur Verfügung gestellt.

