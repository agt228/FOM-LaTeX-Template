\newpage
\section{Umsetzung} \label{sec: Umsetzung}

\subsection{Framing} \label{sec: p_Framing}

Das \textit{betriebswirtschafliche Problem} ist die~\glqq{Vorhersage der Zinsstrukturkurve am Geld- und Kapitalmarkt}~\grqq
% Todo Zitat für "Was ist die Zinskurve" einfügen
Die Zinsstrukturkurve beschreibt im weitesten Sinne den Preis (Zins) des Geldes in Abhängigkeit von der Laufzeit.
~\footcite[\vglf][\pagef]{}
% Todo Bild unterschiedlicher Zinskurven einfügen
\begin{figure}[H]
    \caption{Zinsstrukturkurven an unterschiedlichen Tagen}\label{fig:Zinsstrukturkurven}
    \includegraphics[width=0.9\textwidth]{Zinsstruktur}
    \\
    Quelle: Eigene Darstellung
\end{figure}
In Abbildung~\ref{fig:Zinsstrukturkurven} werden drei verschiedene Zinsstrukturkurven an zufällig gewählten Tagen
abgebildet.
Auf der X-Achse ist die Laufzeit abgebildet und auf der Y-Achse der für die Laufzeit gültige Zinssatz.
Hieraus wird ersichtlich, dass sowohl die Höhe der Zinsen, als auch die Steilheit der Zinskurve variieren.
Darüber hinaus existieren mehrere Zinsstrukturkurven, die unterschieden werden zwischen Ländern und Währungen.
%Todo Bild für Zinssatz im Zeitverlauf einfügen
\begin{figure}[H]
    \caption{Zinssatz für ein Jahr Laufzeit im Zeitverlauf}\label{fig:1Y_Zeitverlauf}
    \includegraphics[width=0.9\textwidth]{1Y_Zeitverlauf}
    \\
    Quelle: Eigene Darstellung
\end{figure}

In Abbildung~\ref{fig:1Y_Zeitverlauf} ist exemplarisch der Zinssatz mit einjähriger Laufzeit im Zeitverlauf abgebildet.
Es ist zu erkennen, dass der Zinssatz einen dynamischen und sich kontinuierlich ändernden Verlauf aufweist und
gewisse Ähnlichkeiten zu Verläufen von beispielweise Aktienkursen aufweist.
Warum die Zinsstruktur eine essenzielle Determinante für die \ac{KD-Bank} darstellt, wird im Nachfolgenden erläutert.
Eine der größten und wichtigsten Positionen in der Gewinn- und Verlustrechnung von Banken und darunter auch der \ac{KD-Bank}
ist das Zinsergebnis.
Die \ac{KD-Bank} generiert ihre Zinserträge sowohl durch die Vergabe von Krediten an ihre Kunden als auch durch die
Investition in zinsbringende Anleihen am Geld- und Kapitalmarkt.
Gleichzeitig werden Zinsaufwendungen an Privatkundeneinleger fällig, die das Geld bei der \ac{KD-Bank} anlegen.
Zusätzlich werden Refinanzierungsgeschäfte getätigt, falls die Höhe der Kundeneinlagen nicht ausreicht, um die
Refinanzierungsmittel zinsbringend anzulegen.
Diese Refinanzierungsgeschäfte werden ebenfalls verzinst, sodass der gesamte Zinsüberschuss der \ac{KD-Bank} die Marge bzw.
die Differenz zwischen den Zinserträgen aus der Mittelverwendung und den Zinsaufwendungen der Mittelherkunft darstellt.
\newline
Des Weiteren wird die Zinsstruktur zur Bewertung der Vermögenswerte der Bank zugrunde gelegt.
Der Vermögenswert der Bank, oder auch als \ac{NPV} bekannt, schwankt also unter anderem durch Änderung Zinsstruktur, die
zur Bewertung herangezogen wird.
Das heißt, dass bei ungünstigen Entwicklungen der Zinsstrukturkurve am Geld- und Kapitalmarkt, der \ac{NPV} der
Bank sinkt.
Insgesamt stellt eine adverse Zinsturkurkurvenentwicklung also ein bedeutendes Risiko für die Bank dar.
Ein weiterer Bereich in der \ac{KD-Bank} bei dem die Zinsstruktur eine wesentliche Determinante für Erfolg / Misserfolg ist,
ist das Treasury.
Das Treasury ist der Bereich in der Bank, der für die kurz-, mittel- und langfristige Finanz- und Liquiditätsplanung zuständig.
Das bedeutet, dass dieser Bereich die Zahlungsströme der Bank analysiert und den \ac{NPV} dieser berechnet.
Hierüber sollen Optimierungsansätze entwickelt werden, bei denen Gewinne maximiert bzw. Verluste begrenzt werden.
Zusammenfassend kann also resümiert werden, dass die Zinsstrukturkurve auf eine Vielzahl von Bereichen der Bank wirkt,
sodass die oben beschriebenen Problembereiche in
\begin{itemize}
    \item Planung des Zinsergebnisses
    \item Risikomanagment
    \item Handel
\end{itemize}
unterteilt werden.
In der Vergangenheit wurden einige Versuche unternommen über diverse Ansätze wie Expertenschätzungen oder quantitative
Analysen die Zinsstrukturkurve zu prognostizieren.
Die Prognosen wurden jedoch häufig um bemerkenswerte Beträge verfehlt, was in den drei oben geschilderten Problembereichen
negative Konsequenzen zu Folge hatte.
Zum Ende eines Geschäftsjahres entstanden demnach hohe Soll-Ist-Abweichungen zwischen dem geplanten und dem tatsächlich
eingetretenen Zinsüberschuss, sodass gleichzeitig die Höhe und die Planung der Gewinnverwendung um die Planabweichung
schwankte.
Durch adverse Zinsentwicklungen, die die \ac{KD-Bank} vorher nicht kommen sehen hat, konnten keine Gegensteuerungsmaßnahmen
eingeleitet werden, was zu zeitweisen Verlusten des \ac{NPV} der Bank führte.
Im Treasury wurden durch unerwartete Zinsentwicklungen teilweise Gewinne liegen gelassen bzw. Verluste realisiert, da
das Portfolio nicht rechtizeitig neu ausgerichtet werden konnte.\\
Für die Operationalisierung des Problems wird kein komplexes Konstrukt als Messgröße verwendet, sondern - wie aus den
oben genannten hervorgeht - die Kennzahl \textit{Gewinn} herangezogen werden, wohlwissend, dass diese Messgröße allein den
intendierten Effekt nicht vollständig abbildet.
Die vollständige Abbildung des intendierten Effektes wird in der Praxis ohnehin selten erreicht.~\footcite[\vglf]
[\pagef 50]{seiter.2019}
Aus Sicht des Autors dieser Hausarbeit werden die erforderlichen Kriterien
\begin{itemize}
    \item Validität
    \item Aktualität
    \item Wirtschaftlichkeit,
\end{itemize}
die an die Messgrößen gestellt werden, erfüllt.
Die Validität liegt insofern vor als dass die Kennzahl \textit{Gewinn} aus Sicht des Autors dieser Arbeit den
intendierten Effekt jedoch großflächig abbildet, sodass diese alleinige Kennzahl hinreichend für die gesamte
Operationalisierung ist.
Dies hat den gleichzeitigen Vorteil, dass keine Gewichtung diverser Kennzahlen vorgenommen werden muss.~\footcite[\vglf]
[\pagef 50]{seiter.2019}
Die Aktualität der Messgröße liegt insofern vor, alsdass aufgezeigt werden kann, wie die Lösung des betriebswirtschaftlichen
Problems sich Rückwirkend betrachtet auf die Messgröße ausgewirkt hätte.
Wäre nämlich in der Vergangenheit die Möglichkeit gegeben gewesen die Zinsstruktur vorherzusagen, dann wären im
Risikomanagement Maßnahmen ergriffen worden, um Verluste zu begrenzen, im Treasury wäre das Portfolio renditebringender
Optimiert werden können und in der Zinsergebnisplanung wären die Planabweichungen geringer gewesen.
Das dritte Qualitätskriterium ist die Wirtschaftlichkeit der Messgröße.~\footcite[\vglf][\pagef 51]{seiter.2019}
Diese liegt dann vor, wenn die Kosten zur Erhebung der Messgröße geringer sind, als der Nutzen.
Da gemäß der Expertenschätzung des Autors dieser Hausarbeit der Nutzen mehrere Millionen Euro beträgt und die Kosten
lediglich mehrere Tausend Euro, um die Auswirkungen der Problemlösung abzuschätzen, betragen, ist die Wirtschaftlichkeit
eindeutig gegeben.\\

%Todo Hier noch etwas zum Relevanznachweis schreiben --> Benchmarking mit der eigenen Vergangenheit, konstante Prognoseannahme

Um das betriebswirtschaftliche Problem weiter zu verfeinern, sei noch zu erwähnen, dass die drei beschriebenen Problemfelder,
auf die die Zinsstrukturkurve wirkt, unterschiedlichen Zeithorizonten zugrunde liegt.
Das heißt, dass eine Prognose der Zinsstrukturkurve unterschiedliche Prognoselängen vorhersagen können muss, um die
Problemstellung bestmöglich zu bewältigen.
Ausgehend vom aktuellen Zeitpunkt t_{ 0 } wird für das Risikomanagement ein Prognosehorizont von fünf Handelstagen benötigt, um
Maßnahmen zur Verlustbegrenzung einzuleiten, wie etwa das Tätigen von derivativen Absicherungsgeschäften.
Im Treasury wird ein Prognosehorizont von 63 Handelstagen benötigt, damit das Portfolio optimiert werden kann.
Für die Planungsrechnung ist bestenfalls die Prognose der Zinsstrukturkurve auf den Zeithorizont eines Geschäftsjahres bzw.
252 Handelstagen erforderlich, damit die Zinsergebnisplanung am treffendsten zum tatsächlichen Eintritt ist.
\newline

Ausgehend von der betriebswirtschaftlichen Problemstellung lässt sich nun das Analytics-Problem ableiten.
Da das Ziel der betriebswirtschaftlichen Problemstellung die Vorhersage der Zinsstrukturkurve ist, eignet sich die
Methode der Predictive Analytics von der Grundidee her am ehesten für die Lösung des Problems.
Da jedes Laufzeitband der Zinsstrukturkurve als eigene Entität gesehen werden kann, sollte bestensfalls für jedes
Laufzeitband ein eigenes Modell entwickelt werden.
Zusätzlich gilt es noch der Anforderung an die unterschiedlichen Zeithorizonte gerecht zu werden, was zur Folge hat,
dass sich der Umfang der Modelle um den Faktor drei ehöhen würde.
Da dies jedoch den Rahmen dieser Hausarbeit sprengen würde, wird sich auf die folgenden vier essenziellen Laufzeitbänder
beschränkt:
\begin{itemize}
    \item 3-Monate
    \item 1-Jahr
    \item 5-Jahre
    \item 10-Jahre
\end{itemize}
Diese sind aus Sicht der \ac{KD-Bank} deshalb die zentralen Laufzeitbänder, als dass viele Berechnungen und Referenzierungen
auf ebendiese Laufzeitbänder abstellen.
% Todo Zitat für "Nelson Siegel" einfügen
Darüberhinaus wurden in der Forschung Ansätze entwickelt, die eine Approximation der gesamten Zinskurve auf Basis der
Zinssätze für ein, fünf und zehn Jahre Laufzeit ermöglichen.~\footcite[\vglf][\pagef]{}
Dies ist ein weiteres Argument dafür sich auf die zuvorgenannten Entitäten zu beschränken.
\newline
Des Weiteren gilt es zu klären welche und wie viele Features sich für die Prognose eignen könnten.
Dies soll vorab über eine explorative Analyse bzw\. über Descriptive Analytics herausgearbeitet werden.

\subsection{Allocation} \label{sec: p_Allocation}
Für die Bearbeitung der im vorherigen Kapitel ausgearbeiteten Problemstellungen werden folgende Ressourcen benötigt:
\begin{itemize}
    \item Daten
    \item IT
    \item Personal
\end{itemize}

Es liegen Daten der Zinsstrukturkurve als Export aus der genossenschaftlichen Risikomanagement-Software VR-Control
vor.
Der Datensatz beeinhaltet tägliche Datensätze mit den jeweils gültigen Zinssätzen für die Laufzeitbänder von einem Tag bis
50 Jahre.
Das Format des Datensatzes ist csv ohne eine feste Struktur, die im Tabellenformat ersichtlich wäre.
\newline
Die Problemstellung wird mittels eigener lokaler Hardware bearbeitet und mittels Python und der für diverse Analytics
Felder zur Verfügung stehenden Bibliotheken wie SK-Learn, Matplotlib, Seaborn, Tensorflow, Keras, Numpy, etc.
\newline
Als Personalressource steht ausschließlich der Autor dieser Hausarbeit zur Verfügung, sodass alle erforderlichen Rollen
lediglich von ihm in Personalunion abgebildet werden.

\subsection{Analytics} \label{sec: p_Analytics}
\begin{itemize}
    \item Datenaufbereitung
    \item Datenanalyse
    \item Evaluation der Ergebnisse
\end{itemize}

\subsection{Preparation} \label{sec: p_Preparation}
\begin{itemize}
    \item Klärung der Mechanismen
    \item Feststellung der Gültigkeitsgrenzen
    \item Visualisierung
\end{itemize}