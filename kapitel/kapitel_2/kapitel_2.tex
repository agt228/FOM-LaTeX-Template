\newpage
\section{Praktische Umsetzung} \label{latexDetails}

In der Pratkischen Umsetzung werden die Grundlagen auf den Anwendungsfall angewendet. Hierfür wird zunächst der \ac{OSMI}-Index erläutert.
Es folgen die Bildklassifizierung und die Umsetzung der Named Entitiy Recognition. Zum Abschluss des praktischen Teils folgt ein Kapitel
mit Consultinganteil.

\subsection{OSMI-Index}

\textcolor{red}{Stichpunkte noch ausformulieren} \\

-Bewertungsinstrument, welches zur Bewertung von E-Commerce Webseiten im Hinblick auf
deren sensorische Gestaltung dient und Verbesserungspotentiale aufzeigt
Parameter die sich nach den Sinnen richten (Haptik, Olfaktorik, Akustik, Gustatorik,
Visualität) \\
-JE Sinn gibt es eine Tabelle die die Evaluierung hinsichtlich bestimmter Indikatoren
vereinfacht ermöglicht. \\
-Indikatoren Bsp je Sinn\\
\begin{itemize}
	\item Haptik: 3D-Bilder, Video, Endowment-Effekt, …
	\item Olfaktorik: Mentale Simulation von Düfte, Schlüsselbegriffe für Düfte
	\item Auditive: Töne/Musik, Sprecher/Stimme, Schlüsselbegriffe für Akustik, …
	\item Gustatorische: Mental Simulation des Schmeckens, Verwendung von
	Farbschemata, Produktoptio \& Oberflächen, …
	\item Visuelle: Farbgebung, dynamische Bilder, Oberfläche, …
\end{itemize}

-Gewichtung der Indikatoren wird im \ac{OSMI} nicht vorgenommen, weil nicht zweifelsfrei zu
argumentieren welcher mehr oder weniger bewertet wird (bisher keine Forschungsarbeiten
dazu) \\
-Indikatoren werden einzeln bewertet \& letztendlich fünf Parameter zw. 0 \& 1 liegen vor.
Daraus ergibt sich Gesamt-Index (ohne Gewichtung, arithmetischer Durchschnitt) für jede
Webseite > auch zw. 0 \& 1 und ist dann der \ac{OSMI}-Index \\
-Je näher der \ac{OSMI} an einer 1, desto erfolgreicher spricht die Webseite die Sensorik an. Wert
nahe 0 erhält wichtige Elemente entsprechend nicht \& erfüllt die Indikatoren nicht \\
-BSP eines \ac{OSMI}-Indexes: Bild einfügen aus der Masterarbeit mymuesli.de oder so



möglicher Aufbau:
1.	Bildklassifizierung
	a.Datenset
	b.Umsetzung
2.	NER
	a.	Datenset
	b.	Umsetzung
3.	Zusammenführung
4.	Consultingteil
