\newpage
\section{Praktische Umsetzung} \label{Praktische Umsetzung}

In der Pratkischen Umsetzung werden die Grundlagen auf den Anwendungsfall angewendet.
Im Rahmen des Consulting-Projektes wurde dabei zu Beginn von der Gruppe ein konkretes Zielbild aufgestellt, welches zum
Projektabschluss umgesetzt werden sollte.
Um den OSMI-Index zu optimieren und zu automatisieren wurde die in Abbildung \textcolor{red}{noch ein Ref Einfügen} Pipeline aufgestellt mithilfe dessen
eine Website in Bezug auf den OSMI verarbeitet wird.
Der erste Schritt stellt dabei eine Vorverarbeitung der zu analysierenden Website dar, indem ein Image Classifier die Haupt- und Unterseiten der Website in
Produkt- und Nicht-Produktseiten unterteilt.
Dies ist aus Sicht der Autoren dieser Hausarbeit ein essenzieller Schritt, da so sichergestellt wird, dass die Bewertung des OSMI-Index lediglich auf Basis
der relevanten URLs vorgenommen wird.
Im darauf folgenden Schritt wird auf die textlichen Inhalte der relevanten Produktseiten ein NLP-Modell angewendet, welches die Entitäten des OSMI-Index erkennt und auszählt.
Im abschließenden Schritt werden die Ergebnisse der Website-Analyse im Rahmen eines Dashboards visualisiert.
Nachfolgend wird die Umsetzung der soeben beschriebenen Schritte im Detail beschrieben und erläutert.

\subsection{Datenvorbereitung}

Sowohl in der Literatur als auch in bereits etablierten, gut entwickelten und veröffentlichten Modellen ist das Gebiet
des Sensory Marketings sehr rar vertreten.
Das bedeutet auch, dass im Hinblick auf den jüngst publizierten OSMI-Index ebenfalls bisher wenig technische Entwicklungen
vorgenommen worden sind.
Zwar existieren in der Literatur diverse NLP-Modelle wie beispielsweise das generelle Spacy-Modell oder der sogenannte BERT
und von beiden erwähnten Modellen Spezialisierungen wie beispielsweise SciSpacy, BioBERT, ESG-BERT, etc\., allerdings ist
bisher kein NLP-Modell auf Marketingkontexte trainiert und publiziert worden, sodass die Notwendigkeit zur Entwicklung eines
eigenen Modells im Kontext des Consulting-Auftrags bestand.
Um im Vorfeld der Website-Bewertung jedoch lediglich nur die relevanten URLs mit Produktinhalt zu selektieren, wurde für den zu
entwickelnden Image Classifier ebenfalls ein von Grund auf neu zu entwickelndes Modell benötigt, da es innerhalb der Literatur
bislang kein derartig existierendes Modell publiziert wurde auf das zurückgegriffen werden konnte.