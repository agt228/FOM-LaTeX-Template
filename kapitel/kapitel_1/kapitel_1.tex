\newpage
\section{Methodik} \label{sec: Methodik}

\subsection{Framing} \label{sec: t_Framing}

Der Business-Analytics Prozess wird eingeleitet durch die Identifikation der zu bearbeitenden Problemstellung.
Dabei werden die folgenden zwei Arten von Problemen unterscheiden:

\begin{enumerate}
    \item Betriebswirschaftliches Problem
    \item Analytics-Problem
\end{enumerate}

Das erst genannte geht dem zweit genannten Problem zeitlich vorweg.
In der Praxis ist es häufig so, dass die Unternehmensumwelt wie beispielsweise Kunden, Lieferanten, etc. auf
betriebswirtschaftlicher Ebene auf Probleme hinweisen.
Alternativ ergeben sich aus dem Unternehmensinneren Impulse, die durch Erfahrungen, zufälligen Beobachtungen, Intuitionen,
usw. der Manager oder auch Mitarbeiter für Problemstellungen, die es zu lösen gilt.~\footcite[\vglf][\pagef 24]{seiter.2019}
Diese Rohfassungen der Problemstellungen gilt es anschließend zu Operationalisieren, um zu überprüfen, ob die Ressourcen,
die zur Lösung des Problems eingesetzt werden sollen, eine gewisse Mindestrelevanz aufweisen.~\footcite[\vglf][\pagef 24]{seiter.2019}
Hierzu werden ein oder mehr quantitaive Messgrößen festgelegt, die den Mehrwert der Problemlösung herausstellen sollen.
Diese Messgrößen müssen die Kriterien Aktualität, Wirtschaftlichkeit, und Validität zwangsweise erfüllen.~\footcite[\vglf][\pagef 24]{seiter.2019}
Aus der Operationalisierung des Problems kann anschließend ein Relevanznachweis erbracht werden.
Dieser stellt den intendierten Effekt der Problemlösung dar.
Gleichzeitig wird ein Relevanzniveau vom Top-Management festgelegt, ab wann eine Problemstellung einen Wichtigkeitsgrad
erreicht hat, um bearbeitet werden zu können.~\footcite[\vglf][\pagef 24]{seiter.2019}
\\
Das Analytics-Problem leitet sich aus dem identifizierten betriebswirtschaftlichen Problem ab.
Hier wird eine erste Grundidee entwickelt, wie das betriebswirtschaftliche Problem gelöst werden kann, bzw.
welche Analytics-Methoden/Algorithmen eingesetzt werden sollen.
Die meisten betriebswirtschaftlichen Problemstellungen lassen sich mit einer oder mehr der drei folgenden
Analytics-Methoden lösen:

\begin{itemize}
    \item Descriptive-Analytics - Identifikation unbekannter Muster
    \item Predictive-Analytics - Konstruktion von Prognosemodellen
    \item Prescriptive-Analytics - Konstruktion von Optimierungsmodellen
\end{itemize}

\subsection{Allocation} \label{sec: t_Allocation}

Nachdem das Problem sowohl auf betriebswirtschaftlicher als auch auf technischer Ebene identifiziert wurde und der
Entschluss gefasst wurde, dass dieses gelöst werden soll, geht es im Teilprozess Allocation darum die notwendigen
Ressourcen bereit zu stellen, die zur Lösung des Problems benötigt werden.~\footcite[\vglf][\pagef 27]{seiter.2019}
Diese Ressourcen können in
\begin{itemize}
    \item Daten
    \item IT und
    \item Personal
\end{itemize}
untergliedert werden.~\footcite[\vglf][\pagef 27]{seiter.2019}

Die Ressource \glqq{Daten}\grqq kann in unterschiedlichen Formaten vorliegen.~\footcite[\vglf][\pagef 27]{seiter.2019}
Der günstigste Fall sind strukturierte Tabellen bzw\. Datenmatritzen, bei der pro Zeile eine Entität beschrieben wird
sowie die Spalten die Attribute der Entität darstellen.~\footcite[\vglf][\pagef 27]{seiter.2019}
Des Weiteren liegen die Daten teilweise in einer Semistruktur wie beispielsweise JSON oder HTML vor, teilweise aber auch
ohne jegliche Struktur wie einfache Texte, Bilder, Videos, etc.~\footcite[\vglf][\pagef 37-38]{cleve.2016}
Diese werden vor der eigentlichen Verarbeitung in eine strukturierte Tabelle überführt.~\footcite[\vglf][\pagef 27]{seiter.2019}
Generell gilt, dass die Daten in einer bestimmten Menge sowie Qualität vorliegen müssen.~\footcite[\vglf][\pagef 27]{seiter.2019}
Wie hoch die die Anforderungen an Menge und Qualität sind, ist von der Komplexität der Problems sowie des zu entwickelnden
Algorithmus abhängig.~\footcite[\vglf][\pagef 27]{seiter.2019}
Es kann also sein, dass bei der Entwicklung der Lösungsmodelle keine hinreichenden Ergebnisse erzielt werden können,
sodass es eine Rückkopplung zum Teilprozess Allocation gibt, um weitere Daten zu beschaffen und damit zielführendere
Ergebnisse zu erzielen.~\footcite[\vglf][\pagef 27]{seiter.2019}\\

Bei der Ressource \glqq{IT}\grqq wird unterschieden zwischen Hardware und Softeware.~\footcite[\vglf][\pagef 28]{seiter.2019}
Die Hardware wird benötigt, um Daten zu speichern und zu verarbeiten, während die Software die Anwendungen beinhaltet,
mit denen die Daten gespeichert, transformiert und visualisiert werden.~\footcite[\vglf][\pagef 28]{seiter.2019}
Die benötigte Leistungsfähigkeit der IT wird von der zu analysierenden Datenmenge determiniert und damit gleichzeitig auch
vom betriebswirtschaftlichen Problem.~\footcite[\vglf][\pagef 29]{seiter.2019}

Da sich Analytics-Probleme zum jetzigen Zeitpunkt nicht vollautomatisiert lösen lassen, wird die Ressource \glqq{Personal}\grqq
benötigt, um den Business Analytics Prozess durchzuführen.~\footcite[\vglf][\pagef 29]{seiter.2019}
Bei dieser Ressource wird im wesentlichen zwischen den folgenden drei Rollen unterschrieden:~\footcite[\vglf][\pagef 29-30]{seiter.2019}
\begin{itemize}
    \item Fachexperten
    \item Analytics Experten
    \item IT Experten
\end{itemize}
Die Fachexperten bringen das notwendige Domänenwissen der ursprünglichen betriebswirtschaftlichen Problemstellung mit
und können mit ihrem Fachwissen besipielsweise bei der Plausibilisierung der Ergebnisse mitwirken.~\footcite[\vglf]
[\pagef 29]{seiter.2019}
Die Analytics Experten sind auf die technische Lösung des Problems spezialisert und entwickeln die benötigten
Algorithmen.~\footcite[\vglf][\pagef 29]{seiter.2019}
Die IT-Experten stellen die benötigte IT-Infrastruktur bereit und sind für die Wartung dieser zuständig.~\footcite[\vglf]
[\pagef 30]{seiter.2019}

Diese drei Rollen können sowohl aus den betriebsinternen Ressourcen stammen als auch durch externe Mitarbeiter ergänzt
werden, da vor allem in kleineren Unternehmen nicht alle Rollen in der benötigten Form angestellt sind.~\footcite[\vglf]
[\pagef 30]{seiter.2019}

\subsection{Analytics} \label{sec: t_Analytics}

In diesem Teilprozess werden Evidenzen aus den Daten gewonnen, um damit das definierte Analytics Problem zu lösen.
~\footcite[\vglf][\pagef 30]{seiter.2019}
Dieser Teilprozess kann in die Teilschritte
\begin{itemize}
    \item Datenaufbereitung
    \item Datenanalyse
    \item Evaluation der Ergebnisse
\end{itemize}
aufgeteilt werden.~\footcite[\vglf][\pagef 30]{seiter.2019}
Im erstgenannten Teilprozess werden die erforderlichen Daten gesammelt, konsolidiert und bereinigt.~\footcite[\vglf]
[\pagef 30]{seiter.2019}
Des Weiteren werden neue Attribute durch Berechnung anhand bestehender Attribute geschaffen, bestehende Attribute eliminiert
oder zusammengeführt.~\footcite[\vglf][\pagef 30]{seiter.2019}
Wie gründlich die Datenaufbereitung geleistet werden muss, wird bedingt durch den im nächsten Teilschritt angewendeten
Algorithmus, denn diese haben spezifische Anforderungen an die Datenaufbereitung.~\footcite[\vglf][\pagef 30]{seiter.2019}
Im Teilschritt der Datenanalyse werden die Evidenzen durch Anwendung von Algorithmen gewonnen.~\footcite[\vglf]
[\pagef 31]{seiter.2019}
Hier werden bedingt durch die Problemstellung unterschiedliche Algorithmen angewendet.~\footcite[\vglf]
[\pagef 31]{seiter.2019}
Für Descriptive Analytics stehen bespielsweise einfache deskriptive statische Maße, Clusteranalysen, Assoziationsanalysen, etc.
zur Verfügung.~\footcite[\vglf][\pagef 31]{seiter.2019}
Bei Predictive Analytics Problemen kommen beispielsweise Regressions-, Klassifikations, oder Zeitreihenanalysen zum
Einsatz.~\footcite[\vglf][\pagef 31]{seiter.2019}
Die Prescriptive Analyse macht sich Optimierungs- oder Simulationsalgorithmen zu Nutze.~\footcite[\vglf][\pagef 31]{seiter.2019}

Im letzten Teilschritt werden dann die Ergebnisse aus dem vorherigen Teilschritt evaluiert.~\footcite[\vglf][\pagef 31]
{seiter.2019}
Je nach eingesetzem Algorithmus kommen auch diverse Evaluationsmaße zum Einsatz.~\footcite[\vglf][\pagef 31]{seiter.2019}
Generell werden die Modelle auf Basis eines Evaluationsdatensatzes oder Validierungsdatensatzes bewertet.~\footcite[\vglf]
[\pagef 31]{seiter.2019}
Dieser wird in der Regel vor der Entwicklung der Modelle von den Trainingsdaten, die zur Entwicklung der Modelle verwendet werden,
herausgelöst.~\footcite[\vglf][\pagef 31]{seiter.2019}
Anhand des Validierungsdatensatzes werden schließlich die Gütemaße berechnet mit denen die Qualität des Modells bewertet
wird.~\footcite[\vglf][\pagef 31]{seiter.2019}

\subsection{Preparation} \label{sec: t_Preparation}

Der finale Teilprozess ist die Preperation. Hier ist das primäre Ziel die gewonnenen Roh-Evidenzen addressatengerecht
aufzubereiten, sodass die Anwender diese in optimaler Weise nutzen können.~\footcite[\vglf][\pagef 31]{seiter.2019}
Die folgenden drei Themenbereiche werden in diesem Teilprozess im Wesentlichen abgehandelt:
\begin{itemize}
    \item Klärung der Mechanismen
    \item Feststellung der Gültigkeitsgrenzen
    \item Visualisierung
\end{itemize}
Die Klärung der Mechanismen beinhaltet die Plausibilisierung der Eingangsdaten im Zusammenhang mit dem Output, um beispielsweise
Scheinzusammenhänge auszuschließen.~\footcite[\vglf][\pagef 31-32]{seiter.2019}
Denn erst wenn die Mechanismen geklärt und für plausibel befunden wurden, kann das Risiko bei Verwendung der Evidenz
minimiert bzw. eliminiert werden.~\footcite[\vglf][\pagef 31-32]{seiter.2019}
Die Klärung der Mechanismen kann durch Detailanalysen, Experimante oder Befragungen der jeweiligen Domänenexperten
abgehandelt werden.~\footcite[\vglf][\pagef 32]{seiter.2019}
Die Feststellung der Gültigkeitsgrenzen ist eine weitere Aufgabe innerhalb dieses Teilprozesses, da eine absolute
Generalisierbarkeit der Evidenzen in der Regel nicht möglich bzw. zulässig ist.~\footcite[\vglf][\pagef 32]{seiter.2019}
Aufgrund angenommener Prämissen, die Menge und Qualität der Daten sowie die verwendeten Algorithmen wird die
Generalisierbarkeit begrenzt, weshalb es unabdingbar ist diese Gültigkeitsgrenzen herauszuarbeiten und zu kommunizieren.
~\footcite[\vglf][\pagef 32]{seiter.2019}
Denn nur so können können Fehlanwendungen der Evidenzen vermieden werden.~\footcite[\vglf][\pagef 32]{seiter.2019}
Im Themenbereich der Visualisierung geht es darum die gewonnen Evidenzen addressatengerecht aufzubereiten, da die
Nutzer häufig von den Analysten abweichen.~\footcite[\vglf][\pagef 32]{seiter.2019}
In diesem Schritt ist es essentiell die Evidenzen verständlich für die Nutzer aufzubereiten.
~\footcite[\vglf][\pagef 32]{seiter.2019}
Dies kann über diverse Visualisierungsformen vorgenommen werden, in die die gewonnen Roh-Evidenzen hineinfließen.
~\footcite[\vglf][\pagef 32]{seiter.2019}
Die Visualisierung ist keine triviale Aufgabe, denn neben den geeigeneten Visualisierungsformen sind auch weitere Komponenten
wie beispielsweise die Farbwahl zu beachten.~\footcite[\vglf][\pagef 32]{seiter.2019}
Insgesamt müssen mit den Visualisierungen vor allem die Rezeptionspräferenzen der Betrachter getroffen werden.~\footcite
[\vglf][\pagef 32]{seiter.2019}

