\section{Fazit}

Das Consulting-Projekt hatte zum Ziel einen Beitrag zur automatisierten und objektiven Quantifizierung des
\ac{OSMI}-Index zu Leisten.
Hierfür wurde das in Kapitel~\ref{subsec_osmi} vorgestellte Zielbild erarbeitet und entsprechend umgesetzt.
Zum Abgabestand der vorliegenden Projektarbeit sind die einzelnen Ansätze und Modelle entwickelt worden.
Es wurde sowohl der Image Classifier zur Identifikation der relevanten Produktseiten programmiert als auch das \ac{NER}-Modell
entwickelt, welches die sensorischen Entitäten aus den Texten heraus extrahieren kann.
Darüber hinaus wurde der Prototyp in Form einer Webpage entworfen, welcher schlussendlich die Ergebnisse der Analyse
zusammenfassen soll und die zu bewertende Website im Vergleich zum Branchenschnitt oder auch zum optimalen \ac{OSMI}-Wert
der Branche darstellt.

Die größte Programmierarbeit des Zielbildes wurde im Rahmen des Consulting-Projektes durch die Autoren dieser Projektarbeit
geleistet.
Die zur Verfügung stehenden Trainings- und Testdaten sind jedoch nicht ausreichend genug gewesen, um präzise Modelle
zu entwickeln.
Das heißt, dass beispielsweise der Image Classifier eine Produktseite von einer Nicht-Produktseite im aktuellen Zustand vergleichbar
mit den Chancen bei einem Münzwurf \glqq{Kopf}\grqq~oder \glqq{Zahl}\grqq~zu erhalten bewertet.
Das \ac{NER}-Modell ist von der Performance vergleichsweise noch schlechter.
Zwar standen der Projektgruppe für beide Modelle ausreichend viele Rohdaten zur Verfügung.
Diese hätten jedoch alle annotiert werden müssen, wofür im Rahmen der Projektdauer wenig zeitliche
Ressourcen vorhanden waren.


Was ist noch offen?
Mehr Trainingsdaten zur Optimierung der Modelle
Ende-zu-Ende Integration der Pipeline, um das Zielbild zu komplementieren
Funktion zur Berechnung des OSMI-Indexes auf Basis unserer Modellinputs

Ausblick  / Entwicklungspotential:
Speech-to-Text-Implementierung, um Sound und Videodateien verarbeiten zu können

In dieser Arbeit wurden Text und Bilder zur automatisierten Berechnung des \ac{OSMI}-Indexes verwendet. Was jedoch nicht berücksichtigt wurde, sind dynamische Medien wie Video und Ton.
Diese sind ebenfalls für den OSMI relevant und könnten daher in zukünftigen Arbeiten betrachtet werden um zur Berechnung des OSMI den wirklich vollständigen Inhalt einer Webseite zu betrachten.