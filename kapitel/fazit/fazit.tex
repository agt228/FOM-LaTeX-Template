\section{Fazit}
Das Consulting-Projekt hatte zum Ziel einen Beitrag zur automatisierten und objektiven Quantifizierung des
\ac{OSMI}-Index zu leisten.
Hierfür wurde das in Kapitel~\ref{subsec_osmi} vorgestellte Zielbild erarbeitet und entsprechend umgesetzt.
Zum Abgabestand der vorliegenden Projektarbeit sind die einzelnen Ansätze und Modelle entwickelt worden.
Es wurde sowohl der Image Classifier zur Identifikation der relevanten Produktseiten programmiert als auch das \ac{NER}-Modell
entwickelt, welches die sensorischen Entitäten aus den Texten heraus extrahieren kann.
Darüber hinaus wurde der Prototyp einer Webpage entworfen, welcher die Ergebnisse der Analyse
für den Endnutzer zugänglich machen soll indem sie die zu bewertende Website im Vergleich zum Branchenschnitt und dem optimalen \ac{OSMI}-Wert
der Branche darstellt.

Die größte Programmierarbeit des Zielbildes wurde im Rahmen des Consulting-Projektes durch die Autoren dieser Projektarbeit
geleistet.
Die zur Verfügung stehenden Trainings- und Testdaten sind jedoch nicht ausreichend genug gewesen, um präzise Modelle
zu entwickeln.
Das heißt in Bezug auf den Image Classifier, dass dieser Produktseiten von Nicht-Produktseiten mit rund 60\% Genauigkeit underscheidet,
was nur unwesentlich besser ist als die Chancen eines Münzwurfes für \glqq{Kopf}\grqq{} oder \glqq{Zahl}\grqq{}.
Das \ac{NER}-Modell ist vergleichsweise sogar noch weniger genau. Es erreicht eine \textit{Accuracy} von \textcolor{red}{15\%},
wodurch es in diesem Zustand noch nicht für einen produktiven Einsatz verwendbar ist. 
Zwar standen der Projektgruppe für beide Modelle ausreichend viele Rohdaten zur Verfügung, jedoch wurden sie nicht alle annotiert. Dafür standen im Rahmen der Projektdauer zu zeitlich zu
wenige Ressourcen zur Verfügung.


\textcolor{red}{
Was ist noch offen?
Mehr Trainingsdaten zur Optimierung der Modelle
Ende-zu-Ende Integration der Pipeline, um das Zielbild zu komplementieren
Funktion zur Berechnung des OSMI-Indexes auf Basis unserer Modellinputs
}
\textcolor{red}{
Ausblick  / Entwicklungspotential:
Speech-to-Text-Implementierung, um Sound und Videodateien verarbeiten zu können
}

Dem Anspruch der vollständig automatisierten Erfassung des \ac{OSMI} konnte die Arbeit ebenfalls nicht ganz gerecht werden. So existieren alle dafür notwendigen Teile wie das Scraping der Webseite,~\footcite[\vglf][]{ostkamp2022a} die Bildklassifizierung und die \acl{NER} des Inhalts sowie die Darstellung der Ergebnisse,~\footcite[\vglf][]{ostkamp2022b}~\footcite[\vglf][]{ostkamp2022c}
jedoch wurden sie nicht miteinander integriert.

In dieser Arbeit wurden Text und Bilder zur automatisierten Berechnung des \ac{OSMI}-Indexes verwendet. Was jedoch nicht berücksichtigt wurde, sind dynamische Medien wie Video und Ton.
Diese sind ebenfalls für den OSMI relevant und könnten daher in zukünftigen Arbeiten betrachtet werden um zur Berechnung des OSMI den vollständigen Inhalt einer Webseite zu betrachten.